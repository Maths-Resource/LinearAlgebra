\documentclass[12pt, a4paper]{article}
\usepackage{natbib}
\usepackage{vmargin}
\usepackage{graphicx}
\usepackage{epsfig}
\usepackage{subfigure}
%\usepackage{amscd}

\usepackage{amssymb}
\usepackage{amsbsy}
\usepackage{amsthm, amsmath}
%\usepackage[dvips]{graphicx}

\renewcommand{\baselinestretch}{1.8}

% left top textwidth textheight headheight % headsep footheight footskip
\setmargins{3.0cm}{2.5cm}{15.5 cm}{23.5cm}{0.5cm}{0cm}{1cm}{1cm}

\pagenumbering{arabic}


\begin{document}

Example 1.1

Find the rank of the matrix 

\bigskip \subsection*{Solution}

Let A= 

Order of A is 2 × 2 $\therefore          \rho(A)\leq 2$

Consider the second order minor



There is a minor of order 2, which is not zero. ∴ $\rho (A) = 2$

%%%%%%%%%%%%%%%%%%%%%%%%%%%%%%%%%%%%%%%%%%%%%%%%5 
\newpage 
\subsection*{Example 1.2}

Find the rank of the matrix 

\bigskip \subsection*{Solution}

Let A= 

Order of A is 2 × 2 ∴\rho(A)\leq 2

Consider the second order minor



Since the second order minor vanishes, $\rho(A) \neq  2$

Consider a first order minor $|−5| \neq  0$

There is a minor of order 1, which is not zero

$\therefore \rho (A) = 1$

%%%%%%%%%%%%%%%%%%%%%%%%%%%%%%%%%%%%%%%%%%%%%%%%5 
\newpage  

Example 1.3

Find the rank of the matrix 

\bigskip \subsection*{Solution}

Let A= 

Order Of A is 3 \times 3

$\therefore \rho (A) \leq 3$

Consider the third order minor  = 6 \neq  0

There is a minor of order 3, which is not zero

∴\rho (A) = 3.

%%%%%%%%%%%%%%%%%%%%%%%%5
\newpage 
 

Example 1.4

Find the rank of the matrix 

\bigskip \subsection*{Solution}

Let A= 

Order Of A is $3 \times 3$

$\therefore \rho (A) \leq 3$

Consider the third order minor 

Since the third order minor vanishes, therefore $\rho(A) \neq  3$

Consider a second order minor 

There is a minor of order 2, which is not zero.

$\therefore \rho(A) = 2.$

%%%%%%%%%%%%%%%%%%%%%%%%%%%%%%%%%%%%%%%%%%%%%%%%%%%%%%%%
\newpage 

\subsection*{Example 1.5}

Find the rank of the matrix 

\bigskip \subsection*{Solution}

Let A = 

Order of A is 3 × 4

$\therefore \rho(A)\leq 3.$

Consider the third order minors



Since all third order minors vanishes, \rho(A) \neq  3.

Now, let us consider the second order minors,

Consider one of the second order minors 

There is a minor of order 2 which is not zero.

∴\rho (A) = 2.




%%%%%%%%%%%%%%%%%%%%%%%%%%%%%%%%%%%%%%%%%%%%%%%%%%%%%%%%%%%%5
\newpage 

%%  -- Echelon form and finding the rank of the matrix (upto the order of 3×4) : Solved Example Problems

Example 1.6

Find the rank of the matrix A= 

\bigskip \subsection*{Solution}

The order of A is 3 × 3.

$\therefore \rho(A) \leq 3.$

Let us transform the matrix A to an echelon form by using elementary transformations.



The number of non zero rows is 2

∴Rank of A is 2.

$\rho (A) = 2.$

Note

A row having atleast one non -zero element is called as non-zero row.

 
%%%%%%%%%%%%%%%%%%%%%%%%%%%%%%%%%%%%%%%%%%%%%%%%%%5
\newpage
Example 1.7

Find the rank of the matrix A= 

\bigskip \subsection*{Solution}

The order of A is 3 × 4.

\therefore \rho (A)\leq 3.

Let us transform the matrix A to an echelon form



The number of non zero rows is 3. \therefore \rho(A) =3.

 

Example 1.8

Find the rank of the matrix A= 

\bigskip \subsection*{Solution}

The order of A is 3 × 4.

\therefore \rho(A) \leq 3.

Let us transform the matrix A to an echelon form



The number of non zero rows is 3.

$\therefore \rho (A) =3.$



Testing the consistency of non homogeneous linear equations (two and three variables) by rank method : Solved Example Problems


%%%%%%%%%%%%%%%%%%%%%%%%%%%%%%%%%%%%%%%%%%%%
\newpage 

Example 1.9

Show that the equations x + y = 5, 2x + y = 8 are consistent and solve them.

\bigskip \subsection*{Solution}

The matrix equation corresponding to the given system is



AX=B



Number of non-zero rows is 2.  

\rho (A )= \rho ([ A, B]) = 2 = Number of unknowns.  

The given system is consistent and has unique solution. 

Now, the given system is transformed into



x + y = 5

y = 2

\therefore (1) ⇒ x + 2 = 5

x = 3

Solution is x = 3, y = 2

%%%%%%%%%%%%%%%%%%%%%%%%%%%%%%%%%%%%%%%%%%%%
\newpage 
 

Example 1.10

Show that the equations 2x + y = 5, 4x + 2 y = 10 are consistent and solve them.

\bigskip \subsection*{Solution}

The matrix equation corresponding to the system is



\rho ( A ) = \rho ([ A, B]) = 1 < number of unknowns

\therefore The given system is consistent and has infinitely many solutions.

Now, the given system is transformed into the matrix equation.



Let us take y = k, k ∈R

⇒ 2x + k = 5

x = 1/2 ( 5 − k)

x = 1/2 ( 5 − k) , y = k for all k ∈R

Thus by giving different values for k, we get different solution. Hence the system has infinite number of solutions.

 
%%%%%%%%%%%%%%%%%%%%%%%%%%%%%%%%%%%%%%%%%%%%
\newpage 

Example 1.11

Show that the equations 3 \times  − 2 y = 6, 6x − 4 y = 10 are inconsistent.

\bigskip \subsection*{Solution}

The matrix equation corresponding to the given system is



∴The given system is inconsistent and has no solution.


%%%%%%%%%%%%%%%%%%%%%%%%%%%%%%%%%%%%%%%%%%%%
\newpage 
 

Example 1.12

Show that the equations 2x + y + z = 5, x + y + z = 4, x − y + 2z = 1 are consistent and hence solve them.

\bigskip \subsection*{Solution}

The matrix equation corresponding to the given system is



Obviously the last equivalent matrix is in the echelon form. It has three non-zero rows.

\rho( A ) = \rho( [A, B] )= 3 = Number of unknowns .

The given system is consistent and has unique solution.

To find the solution, let us rewrite the above echelon form into the matrix form.



x + y + z = 4 (1)

 y + z = 3 (2)

3z = 3 (3)

(3)⇒ z = 1

(2)⇒ y = 3 − z = 2

(1) ⇒ x = 4 − y − z

x=1

\therefore x = 1, y = 2, z = 1

%%%%%%%%%%%%%%%%%%%%%%%%%%%%%%%%%%%%%%%%%%%%
\newpage 
 

Example 1.13

Show that the equations x + y + z = 6, x + 2 y + 3z = 14, x + 4 y + 7z = 30 are consistent and solve them.

\bigskip \subsection*{Solution}

The matrix equation corresponding to the given system is



Obviously the last equivalent matrix is in the echelon form. It has two non-zero rows.

\therefore \rho ( [A, B] ) = 2, \rho ( A) = 2

\rho ( A ) = \rho ( [A, B] ) = 2 < Number of unknowns.

The given system is consistent and has infinitely many solutions.

The given system is equivalent to the matrix equation,



x + y + z = 6  (1)

y + 2z = 8 (2)

(2)⇒ y = 8 − 2z,

(1)⇒ x = 6 − y − z = 6 − (8 − 2 z) − z = z – 2

Let us take z = k, k ∈R , we get x = k − 2, y = 8 − 2k , Thus by giving different values for k we get different solutions. 
Hence the given system has infinitely many solutions.

 

%%%%%%%%%%%%%%%%%%%%%%%%%%%%%%%%%%%%%%%%%%%%
\newpage 

Example 1.14

Show that the equations x − 4 y + 7z = 14, 3 \times  + 8 y − 2z = 13, 7x − 8 y + 26z = 5 are inconsistent.

\bigskip \subsection*{Solution}

The matrix equation corresponding to the given system is



The last equivalent matrix is in the echelon form. [A, B] has 3 non-zero rows and [A] has 2 non-zero rows.



The system is inconsistent and has no solution.

%%%%%%%%%%%%%%%%%%%%%%%%%%%%%%%%%%%%%%%%%%%%
\newpage 
 

Example 1.15

Find k, if the equations x + 2 y − 3z = −2, 3 \times  − y − 2z = 1, 2x + 3y − 5z = k are consistent.

\bigskip \subsection*{Solution}

The matrix equation corresponding to the given system is



For the equations to be consistent, \rho ( [A, B] ) = \rho ( A)= 2

∴21 + 7k = 0

7k = −21 .

k = −3

%%%%%%%%%%%%%%%%%%%%%%%%%%%%%%%%%%%%%%%%%%%%
\newpage 
 

Example 1.16

Find k, if the equations x + y + z = 7, x + 2 y + 3z = 18, y + kz = 6 are inconsistent.

\bigskip \subsection*{Solution}

The matrix equation corresponding to the given system is



For the equations to be inconsistent

\rho ( [A, B] ) \neq  \rho ( A)

It is possible if k − 2 = 0 .

K=2

 

%%%%%%%%%%%%%%%%%%%%%%%%%%%%%%%%%%%%%%%%%%%%
\newpage 
 Example 1.17

Investigate for what values of ‘a’ and ‘b’ the following system of equations

x + y + z = 6, x + 2 y + 3z = 10, x + 2 y + az = b have 

(i) no solution  (ii) a unique solution (iii) an infinite number of solutions.

\bigskip \subsection*{Solution}

The matrix equation corresponding to the given system is



Case (i) For no \bigskip \subsection*{Solution}

The system possesses no solution only when \rho ( A )\neq  \rho ([ A, B]) which is possible only when a − 3 = 0 and b − 10 \neq  0

Hence for a = 3, b \neq  10 , the system possesses no solution.

Case (ii) For a unique \bigskip \subsection*{Solution}

The system possesses a unique solution only when \rho ( A ) = \rho ([ A, B]) = number of unknowns.

i.e when \rho ( A ) = \rho ([ A, B]) = 3

Which is possible only when a − 3 \neq  0 and b may be any real number as we can observe .

Hence for a \neq  3 and b ∈R , the system possesses a unique solution.

Case (iii) For an infinite number of solutions:

The system possesses an infinite number of solutions only when

\rho ( A )= \rho ([ A, B]) < number of unknowns

i,e when \rho ( A)= \rho ([ A, B])= 2 < 3 ( number of unknowns) which is possible only when a − 3 = 0, b − 10 = 0

Hence for a = 3, b =10, the system possesses infinite number of solutions.

 
%%%%%%%%%%%%%%%%%%%%%%%%%%%%%%%%%%%%%%%%%%%%
\newpage 

Example 1.18

The total number of units produced (P) is a linear function of amount of over times in labour (in hours) (l), amount of additional machine time (m) and fixed finishing time (a)

i.e, P = a + bl + cm

From the data given below, find the values of constants a, b and c



Estimate the production when overtime in labour is 50 hrs and additional machine time is 15 hrs.

\bigskip \subsection*{Solution}

We have, P = a + bl + cm

Putting above values we have

6,950 = a + 40b + 10c

6,725 = a + 35b + 9c

7,100 = a + 40b + 12c

The Matrix equation corresponding to the given system is



\therefore The given system is equivalent to the matrix equation



\therefore The production equation is P = 5000 + 30l + 75m

 Pat l = 50, m=15         = 5000 + 30(50) + 75(15)

=7625 units.

∴The production = 7,625 units.


%%%%%%%%%%%%%%%%%%%%%%%%%%%%%%%%%%%%%%%%%%%%%%%%%%%%%%%%%%%%%%%%%%%%%%%%%%%%%%%%%%%%%%%%%%%