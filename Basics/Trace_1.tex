
In linear algebra, the trace of a square matrix A, denoted tr ⁡ ( A ) {\displaystyle \operatorname {tr} (\mathbf {A} )} {\displaystyle \operatorname {tr} (\mathbf {A} )},[1][2] is defined to be the sum of elements on the main diagonal (from the upper left to the lower right) of A.

The trace of a matrix is the sum of its (complex) eigenvalues, and it is invariant with respect to a change of basis. This characterization can be used to define the trace of a linear operator in general. The trace is only defined for a square matrix (n × n). 

%==============================%

Definition

The trace of an n × n square matrix A is defined as[2][3][4]:34

\[ {\displaystyle \operatorname {tr} (\mathbf {A} )=\sum _{i=1}^{n}a_{ii}=a_{11}+a_{22}+\dots +a_{nn}} {\displaystyle \operatorname {tr} (\mathbf {A} )=\sum _{i=1}^{n}a_{ii}=a_{11}+a_{22}+\dots +a_{nn}}\]

where aii denotes the entry on the ith row and ith column of A.
Example

Let A be a matrix, with

\[  {\displaystyle \mathbf {A} ={\begin{pmatrix}a_{11}&a_{12}&a_{13}\\a_{21}&a_{22}&a_{23}\\a_{31}&a_{32}&a_{33}\end{pmatrix}}={\begin{pmatrix}-1&0&3\\11&5&2\\6&12&-5\end{pmatrix}}} {\displaystyle \mathbf {A} ={\begin{pmatrix}a_{11}&a_{12}&a_{13}\\a_{21}&a_{22}&a_{23}\\a_{31}&a_{32}&a_{33}\end{pmatrix}}={\begin{pmatrix}-1&0&3\\11&5&2\\6&12&-5\end{pmatrix}}}\]

Then

\[ {\displaystyle \operatorname {tr} (\mathbf {A} )=\sum _{i=1}^{3}a_{ii}=a_{11}+a_{22}+a_{33}=-1+5+(-5)=-1} {\displaystyle \operatorname {tr} (\mathbf {A} )=\sum _{i=1}^{3}a_{ii}=a_{11}+a_{22}+a_{33}=-1+5+(-5)=-1}\]



%=================================================%
%% - http://mathonline.wikidot.com/the-trace-of-a-square-matrix

Calculating the trace of a matrix is relatively easy. For example, given the following 4×4 matrix A=⎡⎣⎢⎢34−3321−210−2−414375⎤⎦⎥⎥ then tr(A)=3+1+(−4)+5=5.
Example 1

Given the following matrix B, calculate tr(B):
(1)
B=⎡⎣⎢⎢⎢⎢1−4−12232110553533−2242π11203114⎤⎦⎥⎥⎥⎥

We note that there are five entries of the main diagonal, that is b11=1,b22=11,b33=3,b44=1,b55=14. The sum of these entries is the trace of B, that is tr(B)=1+11+3+1+14=30.
Example 2

Find all values of n such that tr(C)=23.
(2)
C=⎡⎣302nn23n2n14⎤⎦

We only give notice to entries in the main diagonal. By the definition of a trace of a matrix, it follows that tr(C)=3+n2+4. We were given that tr(C)=23, and we can therefore solve for n as follows:
(3)
tr(C)=3+n2+423=3+n2+416=n2n=4,−4

Hence n=±4 then tr(C)=23.
Example 3

Prove that if C=A+B, then tr(C)=tr(A)+tr(B) (assume A,B,C are all n×n square matrices).

    Proof: If C=A+B, then:

(4)
C=⎡⎣⎢⎢⎢⎢a11+b11∗∗∗∗a22+b22∗∗∗∗⋱∗∗∗∗ann+bnn⎤⎦⎥⎥⎥⎥

    And therefore we have that:

(5)
tr(C)=(a11+b11)+(a22+b22)+...+(ann+bnn)tr(C)=a11+a22+...+ann+b11+b22+...+bnntr(C)=[a11+a22+...+ann]+[b11+b22+...+bnn]tr(C)=tr(A)+tr(B)■